%%%%%%%%%%%%%%%%%%%%%%%%%%%%%%%%%%%%%%%%%
% a0poster Landscape Poster
% LaTeX Template
% Version 1.0 (22/06/13)
%
% The a0poster class was created by:
% Gerlinde Kettl and Matthias Weiser (tex@kettl.de)
% 
% This template has been downloaded from:
% http://www.LaTeXTemplates.com
%
% License:
% CC BY-NC-SA 3.0 (http://creativecommons.org/licenses/by-nc-sa/3.0/)
%
%%%%%%%%%%%%%%%%%%%%%%%%%%%%%%%%%%%%%%%%%

%----------------------------------------------------------------------------------------
%	PACKAGES AND OTHER DOCUMENT CONFIGURATIONS
%----------------------------------------------------------------------------------------

\documentclass[a0,landscape]{a0poster}

\usepackage{multicol} % This is so we can have multiple columns of text side-by-side
\columnsep=100pt % This is the amount of white space between the columns in the poster
\columnseprule=3pt % This is the thickness of the black line between the columns in the poster

\usepackage[svgnames]{xcolor} % Specify colors by their 'svgnames', for a full list of all colors available see here: http://www.latextemplates.com/svgnames-colors

\usepackage{times} % Use the times font
%\usepackage{palatino} % Uncomment to use the Palatino font

\usepackage{float}
\usepackage{graphicx} % Required for including images
\graphicspath{{figures/}} % Location of the graphics files
\usepackage{booktabs} % Top and bottom rules for table
\usepackage[font=small,labelfont=bf]{caption} % Required for specifying captions to tables and figures
\usepackage{amsfonts, amsmath, amsthm, amssymb} % For math fonts, symbols and environments
\usepackage{wrapfig} % Allows wrapping text around tables and figures
\usepackage[nounderscore]{syntax}
\usepackage{algpseudocode}

\begin{document}

%----------------------------------------------------------------------------------------
%	POSTER HEADER 
%----------------------------------------------------------------------------------------

%TODO center these
\begin{figure}[!htb]
\begin{minipage}[b]{0.19\linewidth}
\includegraphics[height=.8\columnwidth,keepaspectratio]{csdept_logo.png} % Logo or a photo of you, adjust its dimensions here
\end{minipage}
%
\hspace{9cm}
\begin{minipage}[b]{0.59\linewidth}
\veryHuge \color{NavyBlue} \textbf{Curing Sorceror's Apprentice Syndrome} \color{Black}\\ % Title
\Huge\textit{Examining a Real-World Protocol with Denotational Semantics}\\[1cm] % Subtitle
\huge \textbf{Jonathan Wesley Stone}\\ % Author(s)
\huge Advisor: Dr. Tom Turner\\
\huge University of Central Oklahoma Department of Computer Science\\ % University/organization
\end{minipage}
%
\begin{minipage}[b]{0.19\linewidth}
\includegraphics[height=.8\columnwidth,keepaspectratio]{uco_logo.png} % Logo or a photo of you, adjust its dimensions here
\end{minipage}
\end{figure}
%
\vspace{1cm} % A bit of extra whitespace between the header and poster content

%----------------------------------------------------------------------------------------

\begin{multicols}{4} % This is how many columns your poster will be broken into, a poster with many figures may benefit from less columns whereas a text-heavy poster benefits from more

%----------------------------------------------------------------------------------------
%	ABSTRACT
%----------------------------------------------------------------------------------------

\color{Navy} % Navy color for the abstract

\begin{abstract}

The formal semantics of computer languages may be specified operationally, denotationally, or axiomatically. This poster considers the TFTP protocol as a computer language. The protocol's grammar is taken as the language's syntax. Protocol endpoints emit language statements which are interpreted by their remote counterparts. Semantics are understood as defining protocol endpoint behavior. Operational and denotational semantics are modeled and evaluated. The sorceror's apprentice syndrome condition arising from under-specified operational semantics is identified by denotational semantics. The denotational semantic description is modified to correct the syndrome without sacrificing the protocol's functional usefulness.

\end{abstract}

%----------------------------------------------------------------------------------------
%	INTRODUCTION
%----------------------------------------------------------------------------------------

\color{SaddleBrown} % SaddleBrown color for the introduction

\section*{Introduction}

TFTP is a simple (\textit{trivial}) file transfer protocol introduced in 1981 in RFC 783 \cite{Sollins:1983}.  A TFTP session between an initiating client and listening server supports the transmission of one file in either direction, via the client's read or write request. The protocol specifies five packet types which either endpoint is called upon to complementarily issue and handle at various points throughout the transfer. The file is transmitted in packets of 512-byte blocks. The session is terminated when the file transfer is complete.

TFTP is treated here as a formal language whose alphabet consists of packets admitted by the protocol grammar and whose valid strings consist of sequences of those packets. The distinction is drawn between individual packets which are well-formed byte sequences recognized by the protocol grammar, and packet sequences which are recognized by the proposed formal language. A TFTP session between two hosts constitutes an execution of the program. Operational semantics define the algorithmic response of each host to a received packet. Although each host separately receives, recognizes, and interprets individual packets, it is the union of client, server, internetwork, and operational semantics that collectively recognizes this language. This model of TFTP reveals an interesting consequence of RFC 783.

%----------------------------------------------------------------------------------------
%	OBJECTIVES
%----------------------------------------------------------------------------------------

\color{DarkSlateGray} % DarkSlateGray color for the rest of the content

\section*{Main Objectives}

\begin{enumerate}
\item Characterize the protocol TFTP as a computer language.
\item Describe the TFTP language syntax.
\item Describe TFTP's operational semantics as the behavior of states in a state machine.
\item Present a model of TFTP's denotational semantics.
\item Illustrate successful ``execution'' of a TFTP ``program''.
\item Illustrate a problematic execution -- known as Sorceror's Apprentice Syndrome.
\item Modify the denotational semantics to preclude Sorceror's Apprentice Syndrome.
\item Translate the modified denotational semantics back into operational semantics.
\end{enumerate}

%----------------------------------------------------------------------------------------
%	MATERIALS AND METHODS
%----------------------------------------------------------------------------------------

\section*{A Formal Language from TFTP}

TFTP's grammar as defined in RFC 783, in Augmented BNF:
\begin{grammar}
<PAYLOAD> ::= .*

<STR\_MODE> ::= "netascii" / "octet" / "mail"

<FILENAME> :: {\%x01-FF}+

<ERR\_MSG> :: {\%x01-FF}+

<BLOCK\_NUM> ::= \%x00-FF \%x00-FF

<ERROR\_CODE> ::= \%x00-FF \%x00-FF

<PKT\_RRQ> ::= \%x00 \%x01 <FILENAME> \%x00 <STR\_MODE> \%x00

<PKT\_WRQ> ::= \%x00 \%x02 <OP\_WRQ> <FILENAME> \%x00 <STR\_MODE> \%x00

<PKT\_DATA> ::= \%x00 \%x03 <BLOCK\_NUM> <PAYLOAD>

<PKT\_ACK> ::= \%x00 \%x04 <BLOCK\_NUM>

<PKT\_ERROR> ::= \%x00 \%x05 <OP\_ERROR> <ERR\_CODE> <ERR\_MSG> \%x00

<PACKET> ::= <PKT\_RRQ> / <PKT\_WRQ> / <PKT\_DATA> / <PKT\_ACK> / <PKT\_ERROR>
\end{grammar}

Well-formed packets recognized by the above individually form the alphabet of the new grammar, which is annotated with instance parameters $f$ denoting a file of length $n$ blocks indexed by $i$, defined thusly:

\begin{grammar}
<START> ::= $\big[$<PKT\_RRQ>$_{f}$ <PKT\_WRQ>$_{f}\big]$ <LOOP>$_{f,1}$

<DATA\_TIMEOUT$_{f,i}$> ::= <DATA>$_{f,i}$ <LOOP>$_{f,i}$

<ACK\_TIMEOUT$_{f,i}$> ::= <ACK>$_{f,i}$ <LOOP>$_{f,i}$

<DATA$_{f,i}$> ::= <PKT\_DATA>$_{f,i}$ <DATA\_TIMEOUT>$_{f,i}$?

<ACK$_{f,i}$> ::= <PKT\_ACK>$_{f,i}$ <ACK\_TIMEOUT>$_{f,i}$?

<LOOP$_{f,i}$> ::= <DATA>$_{f,i}$ <ACK>$_{f,i}$ <LOOP>$_{f,i+1}$ $\big/$ <STOP>

<STOP> ::= <DATA>$_{f,n}$ <ACK>$_{f,n}$ $\big/$ <PKT\_ERROR>
\end{grammar}

This formalizes the end-point behavior described in RFC 783, namely that a session is initiated with PKT\_RRQ or PKT\_WRQ, and file blocks are sent one at a time via DATA commands which are each acknowledged by a ACK tagged with the block ID. If an endpoint times out awaiting an expected DATA or ACK it assumes this was lost and resends its own last packet by way of requesting retransmission, until all blocks have been transmitted or an error occurs. Figures 1 and 2 illustrate a successful transfer with no errors, and a successful transfer that recovered from a timeout of block 2:

\begin{minipage}{\linewidth}
\centering
  \begin{minipage}{0.45\linewidth}
    \centering
    \begin{figure}[H]
    \centering
      \includegraphics[width=.8\columnwidth,keepaspectratio]{figure2a.png}
      \caption{A successful transfer, recovering from one timeout.}
    \end{figure}
  \end{minipage}
  \begin{minipage}{0.45\linewidth}
    \centering
    \begin{figure}[H]
    \centering
      \includegraphics[width=.8\columnwidth,keepaspectratio]{figure3a.png}
      \caption{A pathological transfer where delay causes duplication.}
    \end{figure}
  \end{minipage}
\end{minipage}

The loop only moves onto the next block when both the DATA and ACK have been transmitted. This provides assurance that all blocks are ultimately transmitted and received by the time the last block has been transmitted. This presents a problem, however. In the unlikely event a particular packet cannot be transmitted (i.e. continually gets lost/dropped), the transmission may never complete and the program theoretically does not terminate. In a more likely scenario, packets can be delayed in transmission beyond the timeout value. Intuitionally, the endpoint need not resend a packet which has been acknowledged, but the current specification does not provide for that exception.
%------------------------------------------------

\subsection*{Mathematical Section}

\begin{grammar}
<START> ::= $\big[$<PKT\_RRQ>$_{f}$ <PKT\_WRQ>$_{f}\big]$ <LOOP>$_{f,1}$

<DATA\_TIMEOUT$_{f,i}$> ::= 

<DATA$_{f,i}$> ::= <DATA\_PKT>$_{f,i}$ <DATA\_TIMEOUT>$_{f,i}$

<LOOP$_{f,i}$> ::= <DATA>$_{f,i}$ <ACK>$_{f,i}$ <LOOP>$_{f,i+1}$ $\big/$ <STOP>

<STOP> ::= <DATA>$_{f,n}$ <ACK>$_{f,n}$ $\big/$ <PKT\_ERROR>
\end{grammar}

The denotational semantics of this language look like this:

\begin{equation}
E = mc^{2}
\label{eqn:Einstein}
\end{equation}

Curabitur mi sem, pulvinar quis aliquam rutrum. (1) edf (2)
, $\Omega=[-1,1]^3$, maecenas leo est, ornare at. $z=-1$ edf $z=1$ sed interdum felis dapibus sem. $x$ set $y$ ytruem. 
Turpis $j$ amet accumsan enim $y$-lacina; 
ref $k$-viverra nec porttitor $x$-lacina. 

Vestibulum ac diam a odio tempus congue. Vivamus id enim nisi:

\begin{eqnarray}
\cos\bar{\phi}_k Q_{j,k+1,t} + Q_{j,k+1,x}+\frac{\sin^2\bar{\phi}_k}{T\cos\bar{\phi}_k} Q_{j,k+1} &=&\nonumber\\ 
-\cos\phi_k Q_{j,k,t} + Q_{j,k,x}-\frac{\sin^2\phi_k}{T\cos\phi_k} Q_{j,k}\label{edgek}
\end{eqnarray}
and
\begin{eqnarray}
\cos\bar{\phi}_j Q_{j+1,k,t} + Q_{j+1,k,y}+\frac{\sin^2\bar{\phi}_j}{T\cos\bar{\phi}_j} Q_{j+1,k}&=&\nonumber \\
-\cos\phi_j Q_{j,k,t} + Q_{j,k,y}-\frac{\sin^2\phi_j}{T\cos\phi_j} Q_{j,k}.\label{edgej}
\end{eqnarray} 

Nulla sed arcu arcu. Duis et ante gravida orci venenatis tincidunt. Fusce vitae lacinia metus. Pellentesque habitant morbi. $\mathbf{A}\underline{\xi}=\underline{\beta}$ Vim $\underline{\xi}$ enum nidi $3(P+2)^{2}$ lacina. Id feugain $\mathbf{A}$ nun quis; magno. Fusce convallis rutrum turpis, quis aliquet enim accumsan id. Vestibulum ullamcorper porttitor convallis. Integer sagittis interdum malesuada. Class aptent taciti sociosqu ad litora torquent per conubia nostra, per inceptos himenaeos. Sed adipiscing tristique orci at ullamcorper. Morbi accumsan, urna et porttitor pulvinar, lacus risus dignissim massa. Proin sollicitudin. Pellentesque eget orci eros. Fusce ultricies, tellus et pellentesque fringilla, ante massa luctus libero, quis tristique purus urna nec nibh.

%----------------------------------------------------------------------------------------
%	RESULTS 
%----------------------------------------------------------------------------------------

\section*{Results}

Donec faucibus purus at tortor egestas eu fermentum dolor facilisis. Maecenas tempor dui eu neque fringilla rutrum. Mauris \emph{lobortis} nisl accumsan. Aenean vitae risus ante. Pellentesque condimentum dui. Etiam sagittis purus non tellus tempor volutpat. Donec et dui non massa tristique adipiscing.
%
\begin{wraptable}{l}{12cm} % Left or right alignment is specified in the first bracket, the width of the table is in the second
\begin{tabular}{l l l}
\toprule
\textbf{Treatments} & \textbf{Response 1} & \textbf{Response 2}\\
\midrule
Treatment 1 & 0.0003262 & 0.562 \\
Treatment 2 & 0.0015681 & 0.910 \\
Treatment 3 & 0.0009271 & 0.296 \\
\bottomrule
\end{tabular}
\captionof{table}{\color{Green} Table caption}
\end{wraptable}
%
Phasellus imperdiet, tortor vitae congue bibendum, felis enim sagittis lorem, et volutpat ante orci sagittis mi. Morbi rutrum laoreet semper. Morbi accumsan enim nec tortor consectetur non commodo nisi sollicitudin. Proin sollicitudin. Pellentesque eget orci eros. Fusce ultricies, tellus et pellentesque fringilla, ante massa luctus libero, quis tristique purus urna nec nibh.

Nulla ut porttitor enim. Suspendisse venenatis dui eget eros gravida tempor. Mauris feugiat elit et augue placerat ultrices. Morbi accumsan enim nec tortor consectetur non commodo. Pellentesque condimentum dui. Etiam sagittis purus non tellus tempor volutpat. Donec et dui non massa tristique adipiscing. Quisque vestibulum eros eu. Phasellus imperdiet, tortor vitae congue bibendum, felis enim sagittis lorem, et volutpat ante orci sagittis mi. Morbi rutrum laoreet semper. Morbi accumsan enim nec tortor consectetur non commodo nisi sollicitudin.

\begin{center}\vspace{1cm}
\includegraphics[width=0.8\linewidth]{placeholder}
\captionof{figure}{\color{Green} Figure caption}
\end{center}\vspace{1cm}

In hac habitasse platea dictumst. Etiam placerat, risus ac.

Adipiscing lectus in magna blandit:

\begin{center}\vspace{1cm}
\begin{tabular}{l l l l}
\toprule
\textbf{Treatments} & \textbf{Response 1} & \textbf{Response 2} \\
\midrule
Treatment 1 & 0.0003262 & 0.562 \\
Treatment 2 & 0.0015681 & 0.910 \\
Treatment 3 & 0.0009271 & 0.296 \\
\bottomrule
\end{tabular}
\captionof{table}{\color{Green} Table caption}
\end{center}\vspace{1cm}

Vivamus sed nibh ac metus tristique tristique a vitae ante. Sed lobortis mi ut arcu fringilla et adipiscing ligula rutrum. Aenean turpis velit, placerat eget tincidunt nec, ornare in nisl. In placerat.

\begin{center}\vspace{1cm}
\includegraphics[width=0.8\linewidth]{placeholder}
\captionof{figure}{\color{Green} Figure caption}
\end{center}\vspace{1cm}

%----------------------------------------------------------------------------------------
%	CONCLUSIONS
%----------------------------------------------------------------------------------------

\color{SaddleBrown} % SaddleBrown color for the conclusions to make them stand out

\section*{Conclusions}

\begin{itemize}
\item Pellentesque eget orci eros. Fusce ultricies, tellus et pellentesque fringilla, ante massa luctus libero, quis tristique purus urna nec nibh. Phasellus fermentum rutrum elementum. Nam quis justo lectus.
\item Vestibulum sem ante, hendrerit a gravida ac, blandit quis magna.
\item Donec sem metus, facilisis at condimentum eget, vehicula ut massa. Morbi consequat, diam sed convallis tincidunt, arcu nunc.
\item Nunc at convallis urna. isus ante. Pellentesque condimentum dui. Etiam sagittis purus non tellus tempor volutpat. Donec et dui non massa tristique adipiscing.
\end{itemize}

\color{DarkSlateGray} % Set the color back to DarkSlateGray for the rest of the content

%----------------------------------------------------------------------------------------
%	FORTHCOMING RESEARCH
%----------------------------------------------------------------------------------------

\section*{Forthcoming Research}

Vivamus molestie, risus tempor vehicula mattis, libero arcu volutpat purus, sed blandit sem nibh eget turpis. Maecenas rutrum dui blandit lorem vulputate gravida. Praesent venenatis mi vel lorem tempor at varius diam sagittis. Nam eu leo id turpis interdum luctus a sed augue. Nam tellus.

 %----------------------------------------------------------------------------------------
%	REFERENCES
%----------------------------------------------------------------------------------------

\nocite{*} % Print all references regardless of whether they were cited in the poster or not
\bibliographystyle{plain} % Plain referencing style
\bibliography{sample} % Use the example bibliography file sample.bib

%----------------------------------------------------------------------------------------
%	ACKNOWLEDGEMENTS
%----------------------------------------------------------------------------------------

\section*{Acknowledgements}

Etiam fermentum, arcu ut gravida fringilla, dolor arcu laoreet justo, ut imperdiet urna arcu a arcu. Donec nec ante a dui tempus consectetur. Cras nisi turpis, dapibus sit amet mattis sed, laoreet.

%----------------------------------------------------------------------------------------

\end{multicols}
\end{document}
