%%%%%%%%%%%%%%%%%%%%%%%%%%%%%%%%%%%%%%%%%
% a0poster Landscape Poster
% LaTeX Template
% Version 1.0 (22/06/13)
%
% The a0poster class was created by:
% Gerlinde Kettl and Matthias Weiser (tex@kettl.de)
% 
% This template has been downloaded from:
% http://www.LaTeXTemplates.com
%
% License:
% CC BY-NC-SA 3.0 (http://creativecommons.org/licenses/by-nc-sa/3.0/)
%
%%%%%%%%%%%%%%%%%%%%%%%%%%%%%%%%%%%%%%%%%

%----------------------------------------------------------------------------------------
%	PACKAGES AND OTHER DOCUMENT CONFIGURATIONS
%----------------------------------------------------------------------------------------

\documentclass[a0,landscape]{a0poster}

\usepackage{multicol} % This is so we can have multiple columns of text side-by-side
\columnsep=100pt % This is the amount of white space between the columns in the poster
\columnseprule=3pt % This is the thickness of the black line between the columns in the poster

\usepackage[svgnames]{xcolor} % Specify colors by their 'svgnames', for a full list of all colors available see here: http://www.latextemplates.com/svgnames-colors

\usepackage{times} % Use the times font
%\usepackage{palatino} % Uncomment to use the Palatino font

\usepackage{float}
\usepackage{graphicx} % Required for including images
\graphicspath{{figures/}} % Location of the graphics files
\usepackage{booktabs} % Top and bottom rules for table
\usepackage[font=small,labelfont=bf]{caption} % Required for specifying captions to tables and figures
\usepackage{amsfonts, amsmath, amsthm, amssymb} % For math fonts, symbols and environments
\usepackage{wrapfig} % Allows wrapping text around tables and figures
\usepackage[nounderscore]{syntax}
\usepackage{algorithm2e}

\begin{document}

%----------------------------------------------------------------------------------------
%	POSTER HEADER 
%----------------------------------------------------------------------------------------

%TODO center these
\begin{figure}[!htb]
\begin{minipage}[b]{0.19\linewidth}
\includegraphics[height=.8\columnwidth,keepaspectratio]{csdept_logo.png} % Logo or a photo of you, adjust its dimensions here
\end{minipage}
%
\hspace{9cm}
\begin{minipage}[b]{0.59\linewidth}
\veryHuge \color{NavyBlue} \textbf{Curing Sorcerer's Apprentice Syndrome} \color{Black}\\ % Title
\Huge\textit{Examining a Real-World Protocol with Denotational Semantics}\\[1cm] % Subtitle
\huge \textbf{Jonathan Wesley Stone}\\ % Author(s)
\huge Advisor: Dr. Tom Turner\\
\huge University of Central Oklahoma Department of Computer Science\\ % University/organization
\end{minipage}
%
\begin{minipage}[b]{0.19\linewidth}
\includegraphics[height=.8\columnwidth,keepaspectratio]{uco_logo.png} % Logo or a photo of you, adjust its dimensions here
\end{minipage}
\end{figure}
%
\vspace{1cm} % A bit of extra whitespace between the header and poster content

%----------------------------------------------------------------------------------------

\begin{multicols}{4} % This is how many columns your poster will be broken into, a poster with many figures may benefit from less columns whereas a text-heavy poster benefits from more

%----------------------------------------------------------------------------------------
%	ABSTRACT
%----------------------------------------------------------------------------------------

\color{Navy} % Navy color for the abstract

\begin{abstract}

The formal semantics of computer languages may be specified operationally, denotationally, or axiomatically. This poster considers the TFTP protocol as a computer language. The protocol's grammar is taken as the language's syntax. Protocol endpoints emit language statements which are interpreted by their remote counterparts. Semantics are understood as defining protocol endpoint behavior. Operational and denotational semantics are modeled and evaluated. The sorcerer's apprentice syndrome condition arising from under-specified operational semantics is identified by denotational semantics. The denotational semantic description is modified to correct the syndrome without sacrificing the protocol's functional usefulness.

\end{abstract}

%----------------------------------------------------------------------------------------
%	INTRODUCTION
%----------------------------------------------------------------------------------------

\color{SaddleBrown} % SaddleBrown color for the introduction

\section*{Introduction}

TFTP is a simple (\textit{trivial}) file transfer protocol introduced in 1981 in RFC 783 \cite{Sollins:1983}.  A TFTP session between an initiating client and listening server supports the transmission of one file in either direction, via the client's read or write request. The protocol specifies five packet types which either endpoint is called upon to complementarily issue and handle at various points throughout the transfer. The file is transmitted in packets of 512-byte blocks. The session is terminated when the file transfer is complete.

TFTP is treated here as a formal language whose alphabet consists of packets admitted by the protocol grammar and whose valid strings consist of sequences of those packets. The distinction is drawn between individual packets which are well-formed byte sequences recognized by the protocol grammar, and packet sequences which are recognized by the proposed formal language. A TFTP session between two hosts constitutes an execution of the program. Operational semantics define the algorithmic response of each host to a received packet. Although each host separately receives, recognizes, and interprets individual packets, it is the union of client, server, internetwork, and operational semantics that collectively recognizes this language. This model of TFTP reveals an interesting consequence of RFC 783.

%----------------------------------------------------------------------------------------
%	OBJECTIVES
%----------------------------------------------------------------------------------------

\color{DarkSlateGray} % DarkSlateGray color for the rest of the content

\section*{Main Objectives}

\begin{enumerate}
\item Characterize TFTP as a computer language.
\item Describe the TFTP language syntax.
\item Describe TFTP's operational semantics as the behavior of states in a state machine.
\item Present a model of TFTP's denotational semantics.
\item Illustrate successful ``execution'' of a TFTP ``program''.
\item Illustrate a problematic execution -- known as Sorcerer's Apprentice Syndrome.
\item Modify the denotational semantics to preclude Sorcerer's Apprentice Syndrome.
\item Translate the modified denotational semantics back into operational semantics.
\end{enumerate}

%----------------------------------------------------------------------------------------
%	MATERIALS AND METHODS
%----------------------------------------------------------------------------------------

\section*{A Formal Language from TFTP}

TFTP's grammar as defined in RFC 783, in Augmented BNF (abbreviated for brevity):
\begin{grammar}
%<PAYLOAD> ::= .*
%
%<STR\_MODE> ::= "netascii" / "octet" / "mail"
%
%<FILENAME> :: {\%x01-FF}+
%
%<ERR\_MSG> :: {\%x01-FF}+
%
%<BLOCK\_NUM> ::= \%x00-FF \%x00-FF
%
%<ERROR\_CODE> ::= \%x00-FF \%x00-FF
%
<PKT\_RRQ> ::= \%x00 \%x01 <FILENAME> \%x00 <STR\_MODE> \%x00

<PKT\_WRQ> ::= \%x00 \%x02 <OP\_WRQ> <FILENAME> \%x00 <STR\_MODE> \%x00

<PKT\_DATA> ::= \%x00 \%x03 <BLOCK\_NUM> <PAYLOAD>

<PKT\_ACK> ::= \%x00 \%x04 <BLOCK\_NUM>

<PKT\_ERROR> ::= \%x00 \%x05 <OP\_ERROR> <ERR\_CODE> <ERR\_MSG> \%x00

<PACKET> ::= <PKT\_RRQ> / <PKT\_WRQ> / <PKT\_DATA> / <PKT\_ACK> / <PKT\_ERROR>
\end{grammar}

Well-formed packets recognized by the above individually form the alphabet of the new grammar, which is annotated with instance parameters $f$ denoting a file of length $n$ blocks indexed by $i$, defined thusly:

\begin{grammar}
<START> ::= $\big[$<PKT\_RRQ>$_{f}$ <PKT\_WRQ>$_{f}\big]$ <LOOP>$_{f,1}$

<DATA\_TIMEOUT$_{f,i}$> ::= <DATA>$_{f,i}$ <LOOP>$_{f,i}$

<ACK\_TIMEOUT$_{f,i}$> ::= <ACK>$_{f,i}$ <LOOP>$_{f,i}$

<DATA$_{f,i}$> ::= <PKT\_DATA>$_{f,i}$ <DATA\_TIMEOUT>$_{f,i}$?

<ACK$_{f,i}$> ::= <PKT\_ACK>$_{f,i}$ <ACK\_TIMEOUT>$_{f,i}$?

<LOOP$_{f,i}$> ::= <DATA>$_{f,i}$ <ACK>$_{f,i}$ <LOOP>$_{f,i+1}$ $\big/$ <STOP>

<STOP> ::= <DATA>$_{f,n}$ <ACK>$_{f,n}$ $\big/$ <PKT\_ERROR>
\end{grammar}

This formalizes the end-point behavior described in RFC 783, namely that a session is initiated with PKT\_RRQ or PKT\_WRQ. File blocks are sent one at a time via DATA commands which are each acknowledged by an ACK tagged with the block ID. If an endpoint times out awaiting an expected DATA or ACK it assumes this will never arrive and resends its own previous packet by way of requesting retransmission, until all blocks have been transmitted or an error occurs. Figure 1 illustrates a transfer that recovers from single packet loss.

\begin{minipage}{\linewidth}
\centering
  \begin{minipage}{0.45\linewidth}
    \centering
    \begin{figure}[H]
    \centering
      \includegraphics[width=.8\columnwidth,keepaspectratio]{figure2a.png}
      \caption{A successful transfer, recovering from one timeout.}
    \end{figure}
  \end{minipage}
  \begin{minipage}{0.45\linewidth}
    \centering
    \begin{figure}[H]
    \centering
      \includegraphics[width=.8\columnwidth,keepaspectratio]{figure3a.png}
      \caption{A pathological transfer where delay causes duplication.}
    \end{figure}
  \end{minipage}
\end{minipage}

The loop only moves onto the next block when both the DATA and ACK have been transmitted. This provides assurance that all blocks are ultimately transmitted and received by the time the last block has been transmitted. This presents a problem, however. In the unlikely event a particular packet cannot be transmitted (i.e. continually gets lost/dropped), the transmission may never complete and the program theoretically does not terminate. In a more likely scenario, packets can be delayed in transmission beyond the timeout value. Intuitionally, the endpoint need not resend a packet which has already been acknowledged, but RFC 738 leaves this decision to the implementor without discussion of the potential issue. When the sender simply treats every ACK as a cue to send the next data block, duplication of all subsequent data block transmissions occurs. When the packet delay rate increases as a function of network utilization, the effects compound and are known as the Sorcerer's Apprentice Syndrome (RFC 1138) (Figure 2).

%------------------------------------------------

\subsection*{Denotational Semantics}

Before and after denotational semantics:

\begin{multicols}{2}
{\small
RFC 738 denotational semantics:
\begin{align*}
%\caption{RFC 738 Denotational Semantics}
M_{R}(R_{f}, s) = &\text{if !exists($f$) then \em{error}} \\
                          &\text{else} \\
                          &\quad \text{set $n$ in $s$ to len($f$)} \\
                          &\quad \text{send $D_{1}$} \\
                          &\quad \text{return $M_{i}$} \\
M_{A}(A_{i}, s) = &\text{if $i$ $<$ VARMAP($n$, $s$) then} \\
                  &\quad \text{send $D_{i+1}$} \\
                  &\text{else} \\
                  &\quad \text{terminate} \\
M_{D}(D_{i}, s) = &\text{send $A_{i}$} \\
M_{T_{a}}(A_{i}, s) = &\text{send $A_{i}$} \\
M_{T_{d}}(D_{i}, s) = &\text{send $D_{i}$} \\
\end{align*}\break
RFC 1138 denotational semantics:
\begin{align*}
%\caption{RFC 738 Denotational Semantics}
M_{R}(R_{f}, s) = &\text{if !exists($f$) then \em{error}} \\
                          &\text{else} \\
                          &\quad \text{set $n$ in $s$ to len($f$)} \\
                          &\quad \text{set $l$ in $s$ to $\emptyset$} \\
                          &\quad \text{send $D_{1}$} \\
                          &\quad \text{return $M_{i}$} \\
M_{A}(A_{i}, s) = &\text{if $i$ $<$ VARMAP($n$, $s$) then} \\
                  &\quad \text{if $i$ $\notin$ VARMAP($l$, $s$) then} \\
                  &\quad \quad \text{append $i$ to $l$ in $s$} \\
                  &\quad \quad \text{send $D_{i+1}$} \\
                  &\quad \text{else} \\
                  &\quad \quad \text{do nothing} \\
                  &\text{else} \\
                  &\quad \text{terminate} \\
M_{D}(D_{i}, s) = &\text{send $A_{i}$} \\
M_{T_{a}}(A_{i}, s) = &\text{send $A_{i}$} \\
M_{T_{d}}(D_{i}, s) = &\text{send $D_{i}$} \\
\end{align*}}
\end{multicols}

It is clear that in the first version, the number of data packets is effectively unbounded, being allowed to vary with the number of timeouts. In the revised denotational semantics, the data packets transmitted is identical to the set of data packets transmitted in an ideal session.

The Sorcerer's Apprentice Syndrome was identified in publication in RFC 1138 (Dec. 1989) after it must have been observed in practice. Although the issue and its fix are intuitively obvious, it would have been caught immediately had the initial design been subjected to a thorough semantic analysis. The practical value of thorough analysis in protocol design is evident.

%----------------------------------------------------------------------------------------
%	REFERENCES
%----------------------------------------------------------------------------------------

\section*{References}

\begin{itemize}
\item{RFC 783.} https://tools.ietf.org/html/rfc783
\item{RFC 1138.} https://tools.ietf.org/html/rfc1138
\item{Concepts of Programming Languages 9e.} Sebesta, Robert W. Addison-Wesley \textcopyright 2009
\item{Programming Language Foundations.} Stump, Aaron. Wiley \textcopyright 2014
\end{itemize}

%----------------------------------------------------------------------------------------

\end{multicols}
\end{document}
